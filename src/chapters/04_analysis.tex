%!TEX root = ../main.tex

\chapter{Analysis}\label{ch:analysis}

%\begin{itemize}
%    \item lr (Language Restrict): 'lang\_en' restricts the search results to pages in English.
%    \item gl (Geolocation): 'us' sets the country of origin for the search to the United States, affecting localized results.
%    \item hl (Host Language): 'en' sets the interface language to English.
%    \item num: requests 100 search results per page (though Google may not always return this many).
%\end{itemize}



\section{A section}

\tikzset{vertex style/.style={
    draw=#1,
    thick,
    fill=#1!70,
    text=white,
    ellipse,
    minimum width=2cm,
    minimum height=0.75cm,
    font=\small,
    outer sep=3pt,
  },
  text style/.style={
    sloped,
    text=black,
    font=\footnotesize,
    above
  }
}

\begin{figure}[ht]
    \centering
    \begin{tikzpicture}[node distance=2.75cm,>={Stealth[]}]
        \node[vertex style=cyan] (Rk) {Righteous Kill};
        \node[vertex style=orange, above of=Rk,xshift=2em] (BD) {Bryan Dennehy}
        edge [<-,cyan!60!blue] node[text style,above]{starring} (Rk);
    \end{tikzpicture}
    
    \caption{Image created with TikZ} \label{fig:T1}
\end{figure}

\paragraph{}
Lorem ipsum dolor sit amet, consectetur adipiscing elit, sed do eiusmod tempor incididunt ut labore et dolore magna aliqua. Ut enim ad minim veniam, quis nostrud exercitation ullamco laboris nisi ut aliquip ex ea commodo consequat. Duis aute irure dolor in reprehenderit in voluptate velit esse cillum dolore eu fugiat nulla pariatur. Excepteur sint occaecat cupidatat non proident, sunt in culpa qui officia deserunt mollit anim id est laborum.

\begin{lstlisting}[language=Python, caption=Code snippet example]
import numpy as np
    
def incmatrix(genl1,genl2):
    m = len(genl1)
    n = len(genl2)
    M = None #to become the incidence matrix
    VT = np.zeros((n*m,1), int)  #dummy variable

    test = "String"
    
    #compute the bitwise xor matrix
    M1 = bitxormatrix(genl1)
    M2 = np.triu(bitxormatrix(genl2),1) 

    for i in range(m-1):
        for j in range(i+1, m):
            [r,c] = np.where(M2 == M1[i,j])
            for k in range(len(r)):
                VT[(i)*n + r[k]] = 1;
                VT[(i)*n + c[k]] = 1;
                VT[(j)*n + r[k]] = 1;
                VT[(j)*n + c[k]] = 1;
                
                if M is None:
                    M = np.copy(VT)
                else:
                    M = np.concatenate((M, VT), 1)
                
                VT = np.zeros((n*m,1), int)
    
    return M
\end{lstlisting}