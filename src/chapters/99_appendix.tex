\begin{appendices}
    \chapter{Prompt Templates}\label{ch:prompt-templates}
    In this section, we present the prompt templates used in the pipeline.
    \section{Human-understandable text generation Prompt}\label{sec:prompt-templates:human-understandable}
    \begin{Verbatim}[fontsize=\small, frame=single, label={Prompt template for generating human-readable text}]
Task Description:
Convert a kg triple into a meaningful human readable sentence.

Instructions:
    Given a subject, predicate, and object from a kg, form a
    grammatically correct and meaningful sentence that conveys
    the relationship between them.

Examples:
Input:
    Subject: Alexander_III_of_Russia
    Predicate: isMarriedTo
    Object:  Maria_Feodorovna__Dagmar_of_Denmark_
    Output: {"output" : "Alexander III of Russia is married to Maria
                        Feodorovna, also known as Dagmar of Denmark."}

Input:
    Subject: Quentin_Tarantino
    Predicate: produced
    Object: From_Dusk_till_Dawn
    Output: {"output": "Quentin Tarantino produced the film
                        From Dusk till Dawn."}

Input:
    Subject: Joseph_Heller
    Predicate: created
    Object: Catch-22
    Output: {"output": "Joseph Heller created the novel Catch-22."}

Do the following:
Input:
Subject: {knowledge_graph.subject}
Predicate: {knowledge_graph.predicate}
Object: {knowledge_graph.object}
The output should be a JSON object with the key "output" and
the value as the sentence. The sentence should be human-readable
and grammatically correct. The subject, predicate, and object
can be any valid string without having extra information.
    \end{Verbatim}
    \section{Question Generation Prompt}\label{sec:prompt-templates:10-question}
    \begin{Verbatim}[fontsize=\small, frame=single, label={Prompt template for generating 10 questions for each triple}]
You are an intelligent system with access to a vast amount of
information. I will provide you with a knowledge graph in the
form of triples (subject, predicate, object).

Your task is to generate ten questions based on the kg.
The questions should assess understanding and insight into the
information presented in the graph.

Provide the output in JSON format, with each question having a unique
identifier. Instructions:
 1.Analyze the provided knowledge graph.
 2.Generate ten questions that are relevant to the information in kg.
 3.Provide the questions in JSON format, each with a unique identifier.

Input Knowledge Graph: Albert Einstein bornIn Ulm, Germany
Expected Response: {
  "questions": [
    {"id": 1,
    "question": "Where was Albert Einstein born?"},
    {"id": 2,
    "question": "What is Albert Einstein known for?"},
    {"id": 3,
    "question": "In what year was the Theory of Relativity published?"},
    {"id": 4,
    "question": "Where did Albert Einstein work?"},
    {"id": 5,
    "question": "What prestigious award did Albert Einstein win?"},
    {"id": 6,
    "question": "Which theory is associated with Albert Einstein?"},
    {"id": 7,
    "question": "Which university did Albert Einstein work at?"},
    {"id": 8,
    "question": "What did Albert Einstein receive the Nobel Prize in?"},
    {"id": 9,
    "question": "In what field did Albert Einstein win a Nobel Prize?"},
    {"id": 10,
    "question": "Name the city where Albert Einstein was born."}
]}
Considering the above information, please respond to this kg: {query}
The output should be in JSON format with each question having a unique
identifier and question doesn't contain term knowledge graph, without
any additional information
    \end{Verbatim}

\chapter{Chunking Strategies}\label{ch:chunking}
    As discussed in Section~\ref{sec:chunking-strategies}, the method used to chunk input text is a critical decision in the design of a \ac{RAG} system.
    Here, we present concrete examples of how different chunking strategies affect the segmentation of text, using the \textit{correct\_spouse\_00134} entry from the FactBench dataset.
    We report the best node found by through our pipeline for each chunking strategy.
    \section{Text Splitter - Chuck Size 512}\label{sec:chunking:text-splitter}

    \section{Small2Big}\label{sec:chunking:small2big}

    \section{Sliding Window - Window Size 3}\label{sec:chunking:sliding-window}
    The knowledge graph triple \textit{correct\_award\_00000} from the FactBench dataset with triple "Henry Dunant award Nobel Peace Prize" is used as an example.
    The model used for this example is \textit{Gemma2} with \textit{similarity\_top\_k} set to 3, and \textit{BAAI/bge-small-en-v1.5} as embedding model.
    The documents are selected using \textit{ms-marco-MiniLM-L-6-v2} discussed in~\ref{subsec:supervised-methods}.
    \begin{table}[h!]
        \noindent
        \resizebox{\textwidth}{!}{
            \begin{tabular}{l}
                \toprule
                \textbf{Window} - Highlighted Text is Original Text \\
                \midrule
                \shortstack[l]{
                    You can read more about that here: From the first Nobel Prize award ceremony, 1901 \\
                    The announcement that the founder of the Red Cross had been chosen as Peace Prize \\
                    laureate met with mixed reactions. Dunant had been awarded the prize for ameliorating \\
                    the suffering of wounded soldiers, not for organising peace congresses or reducing  \\
                    standing forces, as stipulated in Alfred Nobel’s will. The Nobel Committee had chosen \\
                    a broad interpretation of the provision that a laureate should “further fraternity \\
                    between nations”. \colorbox{pink}{The Red Cross: three-time recipient of the Peace Prize Henry Dunant} \\
                    \colorbox{pink}{(1828–1910).}  Switzerland, “for his humanitarian efforts to help wounded soldiers and \\
                    create international understanding” Frédéric Passy (1822–1912).  France, “for his \\
                    lifelong work for international peace conferences, diplomacy and arbitration.”
                } \\ \hline

                \shortstack[l]{
                    On 10th of December 1901 the first Nobel Peace Prize was awarded. It went to Henry Dunant, \\
                    founder of the International Committee of the Red Cross, who shared the first Nobel Peace \\
                    Prize with Frédéric Passy, a leading international pacifist of the time. \\
                    \colorbox{pink}{Since then, the Red Cross has been awarded the Peace Prize three times.} \\
                    The Red Cross: Three-time recipient of the Peace Prize Four of them given out in Stockholm  \\
                    and one, the Peace Prize, in Christiania, as Oslo was then called. You can read more about \\
                    that here: From the first Nobel Prize award ceremony, 1901 The announcement that the founder\\
                    of the Red Cross had been chosen as Peace Prize laureate met with mixed reactions.\\
                    Dunant had been awarded the prize for ameliorating the suffering of wounded soldiers, not for \\
                    organising peace congresses or reducing standing forces, as stipulated in Alfred Nobel’s will.
                } \\ \hline
                \shortstack[l]{
                    Henry Dunant \\
                    The Nobel Peace Prize 1901 \\
                    Nobel co-recipient: Frédéric Passy \\
                    Role: Founder of the International Committee of the Red Cross, Geneva, Originator Geneva \\
                    Convention (Convention de Genève) Nobel Prize Cash and Philanthropy \\
                    Jean Henry Dunant, though poor, donated his Nobel Prize money to charity. Hans Daae, a \\
                    military physician, managed to get the money deposited in a bank in Norway. \colorbox{pink}{Thus Dunant’s}\\
                    \colorbox{pink}{creditors could not claim the money.} When Dunant was alive the money remained untouched in \\
                    the bank. He lived frugally in a Swiss nursing home. Dunant’s will bequeathed one half of \\
                    the money to the Norwegian Red Cross and the Norwegian Women’s Public Health Association.
                } \\
                \bottomrule
            \end{tabular}}\caption{Sliding Window - Window Size 3}\label{tab:table-sliding-window}
        \label{tab:table}
    \end{table}


\end{appendices}