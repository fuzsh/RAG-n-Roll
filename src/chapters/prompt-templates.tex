\chapter{Prompt Templates}\label{ch:prompt-templates}
In this section, we present the prompt templates used in the pipeline.
\section{Human-understandable text generation Prompt}\label{sec:prompt-templates:human-understandable}
\begin{Verbatim}[fontsize=\small, frame=single, label={Prompt template for generating human-readable text}]
Task Description:
Convert a kg triple into a meaningful human readable sentence.

Instructions:
    Given a subject, predicate, and object from a kg, form a
    grammatically correct and meaningful sentence that conveys
    the relationship between them.

Examples:
Input:
    Subject: Alexander_III_of_Russia
    Predicate: isMarriedTo
    Object:  Maria_Feodorovna__Dagmar_of_Denmark_
    Output: {"output" : "Alexander III of Russia is married to Maria
                        Feodorovna, also known as Dagmar of Denmark."}

Input:
    Subject: Quentin_Tarantino
    Predicate: produced
    Object: From_Dusk_till_Dawn
    Output: {"output": "Quentin Tarantino produced the film
                        From Dusk till Dawn."}

Input:
    Subject: Joseph_Heller
    Predicate: created
    Object: Catch-22
    Output: {"output": "Joseph Heller created the novel Catch-22."}

Do the following:
Input:
Subject: {knowledge_graph.subject}
Predicate: {knowledge_graph.predicate}
Object: {knowledge_graph.object}
The output should be a JSON object with the key "output" and
the value as the sentence. The sentence should be human-readable
and grammatically correct. The subject, predicate, and object
can be any valid string without having extra information.
\end{Verbatim}


\section{Question Generation Prompt}\label{sec:prompt-templates:10-question}
\begin{Verbatim}[fontsize=\small, frame=single, label={Prompt template for generating 10 questions for each triple}]
You are an intelligent system with access to a vast amount of
information. I will provide you with a knowledge graph in the
form of triples (subject, predicate, object).

Your task is to generate ten questions based on the kg.
The questions should assess understanding and insight into the
information presented in the graph.

Provide the output in JSON format, with each question having a unique
identifier. Instructions:
 1.Analyze the provided knowledge graph.
 2.Generate ten questions that are relevant to the information in kg.
 3.Provide the questions in JSON format, each with a unique identifier.

Input Knowledge Graph: Albert Einstein bornIn Ulm, Germany
Expected Response: {
  "questions": [
    {"id": 1,
    "question": "Where was Albert Einstein born?"},
    {"id": 2,
    "question": "What is Albert Einstein known for?"},
    {"id": 3,
    "question": "In what year was the Theory of Relativity published?"},
    {"id": 4,
    "question": "Where did Albert Einstein work?"},
    {"id": 5,
    "question": "What prestigious award did Albert Einstein win?"},
    {"id": 6,
    "question": "Which theory is associated with Albert Einstein?"},
    {"id": 7,
    "question": "Which university did Albert Einstein work at?"},
    {"id": 8,
    "question": "What did Albert Einstein receive the Nobel Prize in?"},
    {"id": 9,
    "question": "In what field did Albert Einstein win a Nobel Prize?"},
    {"id": 10,
    "question": "Name the city where Albert Einstein was born."}
]}
Considering the above information, please respond to this kg: {query}
The output should be in JSON format with each question having a unique
identifier and question doesn't contain term knowledge graph, without
any additional information
\end{Verbatim}